% !TEX TS-program = pdflatex
% !TEX encoding = UTF-8 Unicode

% This is a simple template for a LaTeX document using the "article" class.
% See "book", "report", "letter" for other types of document.

\documentclass[11pt]{article} % use larger type; default would be 10pt

\usepackage[utf8]{inputenc} % set input encoding (not needed with XeLaTeX)

%%% Examples of Article customizations
% These packages are optional, depending whether you want the features they provide.
% See the LaTeX Companion or other references for full information.

%%% PAGE DIMENSIONS
\usepackage{geometry} % to change the page dimensions
\geometry{a4paper} % or letterpaper (US) or a5paper or....
% \geometry{margin=2in} % for example, change the margins to 2 inches all round
% \geometry{landscape} % set up the page for landscape
%   read geometry.pdf for detailed page layout information

\usepackage{graphicx} % support the \includegraphics command and options

% \usepackage[parfill]{parskip} % Activate to begin paragraphs with an empty line rather than an indent

%%% PACKAGES
\usepackage{booktabs} % for much better looking tables
\usepackage{array} % for better arrays (eg matrices) in maths
\usepackage{paralist} % very flexible & customisable lists (eg. enumerate/itemize, etc.)
\usepackage{verbatim} % adds environment for commenting out blocks of text & for better verbatim
\usepackage{subfig} % make it possible to include more than one captioned figure/table in a single float
% These packages are all incorporated in the memoir class to one degree or another...

%%% HEADERS & FOOTERS
\usepackage{fancyhdr} % This should be set AFTER setting up the page geometry
\pagestyle{fancy} % options: empty , plain , fancy
\renewcommand{\headrulewidth}{0pt} % customise the layout...
\lhead{}\chead{}\rhead{}
\lfoot{}\cfoot{\thepage}\rfoot{}

%%% SECTION TITLE APPEARANCE
\usepackage{sectsty}
\allsectionsfont{\sffamily\mdseries\upshape} % (See the fntguide.pdf for font help)
% (This matches ConTeXt defaults)

%%% ToC (table of contents) APPEARANCE
\usepackage[nottoc,notlof,notlot]{tocbibind} % Put the bibliography in the ToC
\usepackage[titles,subfigure]{tocloft} % Alter the style of the Table of Contents
\renewcommand{\cftsecfont}{\rmfamily\mdseries\upshape}
\renewcommand{\cftsecpagefont}{\rmfamily\mdseries\upshape} % No bold!

%%% END Article customizations

%%% The "real" document content comes below...

\title{Atmosphere and climate 2017}
\author{Ministry for the Environment and Statistics New Zealand}
%\date{} % Activate to display a given date or no date (if empty),
         % otherwise the current date is printed 

\begin{document}
\maketitle
The atmosphere is the layer of gases that surrounds Earth. Climate is the pattern in variables such as temperature, humidity, wind, and precipitation. Industrial, agricultural, and other human activities emit gases that affect the atmosphere and climate.

\section{Overview}
This chapter sets out why we care about the condition of our atmosphere and climate, and presents information on the pressures, state, and trends of the atmosphere and climate in New Zealand.

\subsection{Our atmosphere and climate – a summary}
New Zealand has a temperate climate that supports our natural environment, economy, and way of life. Our climate is naturally variable because of our location in the South Pacific Ocean and our small but mountainous land area. These factors contribute to the extreme weather, such as heavy rainfall and droughts, which often lead to significant economic and social losses.

Although naturally variable, our climate is also changing. The biggest driver of change is the increase in greenhouse gases from human activities, which is causing temperatures to rise. Global net emissions of greenhouse gases have risen 33 percent since 1990. Between 1990 and 2011, New Zealand emitted around 0.1 percent of global emissions. Carbon dioxide is the greenhouse gas that has the greatest impact over the long term with concentrations over New Zealand rising about 21 percent since 1972. Some changes in our climate, glaciers, and oceans are linked to increasing concentrations of greenhouse gases in our atmosphere. New Zealand’s temperature increased around 0.9 degrees Celsius in the past 100 years. Glacier mass in New Zealand decreased about 36 percent since 1978, while sea levels around New Zealand rose between 1.3 and 2.1 millimetres on average annually over the last century.

Rainfall and wind patterns are also expected to change as a result of the ongoing increase in greenhouse gases. However, because our weather is highly variable year-to-year, it is difficult to detect these effects on our climate. So far, we see no clear trends in the data we have analysed for rainfall and extreme weather patterns (eg lightning strikes and three-day rainfall).

A major feature of our atmosphere and climate is our relatively high levels of ultraviolet (UV) light for our latitude. New Zealand has one of the highest rates of skin cancer (melanoma) incidence in the world partly because of our exposure to high levels of UV light. There has been no discernible trend in the total incidence rate of melanoma since 1996.

\section{Why the condition of our atmosphere and climate is important}
The condition of our atmosphere and climate is important for many reasons. The following section provides information on economic effects, water availability, changes in glacier extent, and health effects.

\subsection{Economic effects}
Our dependence on our climate means that extreme weather has a significant impact on our economy – especially our agricultural sector, which is reliant on a temperate climate. Although extreme weather events have always occurred, these are expected to become more frequent due to climate change (NIWA, 2009).

Extreme weather events can cause millions of dollars of damage (Insurance Council of New Zealand (ICNZ), 2014) and lead to injury or the loss of life. Extreme weather includes storms that cause flooding or wind damage, droughts, and snowstorms.

In 2014, the ICNZ identified eight extreme weather events - mainly storms and floods - that caused over \$ 150 million in insurance costs for damage to homes, businesses, and farms (ICNZ, 2014) - though the total costs are likely to be much higher. For example, the North Island floods of 2004 cost more than \$ 140 million in insurance payouts, but the full cost of damage was an estimated \$ 355 million. About half (around \$ 185 million) was for damage to agriculture (Horizons Regional Council, 2004).

Droughts are costly to the agricultural sector. The 2013 drought contributed to an estimated 4.4 percent decrease in milk production per cow (DairyNZ, 2014). Rising temperatures and changes in rainfall patterns are expected to make drought more frequent and more intense, especially in the east and north of the country (Reisinger et al, 2014).

Changes in climate affect the tourism industry. For example, rising average temperatures could reduce snowfall, and in turn the number of days that ski fields operate.

Our cultural sites are vulnerable to climate change. Many sites significant to Māori are in low-lying or coastal areas, which may be affected by sea-level rise or increased coastal-storm activity (Reisinger et al, 2014).

For more detail see Environmental indicators Te taiao Aotearoa: Insurance losses for extreme weather events and Ski-field operating days.

\subsection{Water availability}
Climate influences the amount of water in our lakes, rivers, streams, and aquifers, and therefore how much water is available for our use.

Climate affects rainfall and other forms of precipitation. The average annual outflow to sea is around 80 percent of the volume of precipitation. Evapotranspiration, or the loss of water to the atmosphere by evaporation, and from plants by transpiration, accounts for most of the remaining water from precipitation.

As well as the water that flows out to sea through New Zealand’s rivers and streams (see figure 10), a large amount of water is stored in the ground (in aquifers). An estimated total of 711 billion cubic metres is held in aquifers, about three-quarters of this in Canterbury.

%Figure 10
%Read a description of this image

From current data, we cannot determine whether climate change has had an effect on surface or groundwater.

For more detail see Environmental indicators Te taiao Aotearoa: Water physical stocks: precipitation and evapotranspiration.

\subsection{Glacier extent}
The volume of glacier ice decreased 36 percent since 1978 (see figure 11). Increased temperatures over the last century have caused glaciers to melt. If temperatures continue to rise, it is likely that the extent of glaciers will shrink further.

%Figure 11
%Read a description of this image

\subsection{Health effects}
Climate, and change in climate, can affect our health. This section provides information on the incidence of melanoma (skin cancer), and the impact climate change may have on the incidence of some food- and water-borne illnesses.

\textbf{High ultraviolet levels affect our health}
New Zealand has one of the highest rates of skin cancer (melanoma) in the world (Ministry of Health, 2015). This is linked to the high ultraviolet (UV) light in our atmosphere, with peak intensities 40 percent greater than comparable latitudes in Europe. This is partly due to Earth’s orbit bringing the Southern Hemisphere closer to the sun in summer. Air clarity or the effects of cloud also affect our UV levels (NIWA, 2015).

The ‘ozone hole’ does not have a significant impact on UV levels in our atmosphere as it occurs south of New Zealand. While ozone levels over New Zealand decreased slightly since 1978, our UV levels have not shown a consistent trend across monitoring sites.

There has been no discernible trend in the overall incidence of skin cancer since 1996, but there has been a statistically significant increasing trend for males. Long lag effects between exposure to UV light and the occurrence of skin cancer means any recent changes in exposure are not evident in our current skin cancer data.

Our high rates of melanoma are also linked to the relatively high level of exposure New Zealanders have to the sun, and the large number of fair-skinned people in New Zealand. Relative to population, New Zealand is estimated to have the highest incidence and death rate from melanoma in the world (Ferlay et al, nd). In 2011, 2,204 people were reported as having melanoma, while 359 died from it (Ministry of Health, 2014).

The incidence of cancer is strongly influenced by age. New Zealand has an increasing proportion of people in the older age groups so we can expect an increase in the incidence of melanoma. For this reason, we use ‘age-standardised’ data to compare data over time periods and between countries. The age-standardised incidence of melanoma in New Zealand between 1996 and 2013 showed no clear trend (see figure 12).

%Figure 12
%Read a description of this image

For more detail see Environmental indicators Te taiao Aotearoa: Occurrence of skin cancer.

\textbf{Rising temperatures may affect our health}

International research suggests rising temperatures increase the incidence of some diseases, such as food-borne diseases caused by Campylobacter and Salmonella, and water-borne diseases caused by Cryptosporidium (Health Analysis and Information for Action, 2015; Lal et al, 2013). However, the incidence of other illnesses more common in colder months, such as influenza, may decrease as a result of rising temperatures (Tompkins et al, 2012).

For more detail see Environmental indicators Te taiao Aotearoa: Food - and water-borne diseases and Influenza.

\section{The pressures on out atmosphere and climate}

Many factors affect the atmosphere and climate – both human-caused and natural. This section presents our latest findings on human-caused pressures (global emissions of greenhouse gases and ozone-depleting substances) and natural pressures (ocean conditions and climate oscillations).

\subsection{Human-caused pressures}
This section sets out the main human-caused pressures on the atmosphere: emissions of greenhouse gases and ozone-depleting substances.

\textbf{Glomabl emissions of greenhouse gases are still increasing}
Human activities – mainly the burning of fossil fuels – have led to increased concentrations of greenhouse gases in the atmosphere. Between 1990 and 2011, global net emissions rose 33 percent (see figure 13).

Over this period, New Zealand emitted around 0.1 percent of global emissions. By comparison, the United States contributed an average of 16 percent, and China 15 percent.

%Figure 13
%Note: The United Nations Framework Convention on Climate Change is the preferred data source, but because some countries do not report on all years since 1990, the data are supplemented with Climate Analysis Indicators Tool data. The effect of the various greenhouse gases are combined into one indicator, known as CO2 (carbon dioxide) equivalent.
%Read a description of this image


This graph shows the estimated net emissions of greenhouse gases by major emitting countries (United States of America, Russia, rest of the world, Japan, Indonesia, India, European Union, People’s Republic of China, and Brazil) between 1990 and 2011. Visit the MfE data service for the full breakdown of the data.

New Zealand’s net greenhouse gas emissions increased 42 percent between 1990 and 2013 (see figure 14). Emissions from agriculture was the largest contributor at 48 percent (mainly methane emissions from cattle and sheep), followed by emissions from energy production, at 39 percent.

%Figure 14
%Note: Mt C02 equivalent – megatonnes equivalent carbon dioxide.
%Read a description of this image

For more detail see Environmental indicators Te taiao Aotearoa: Global greenhouse gas emissions and New Zealand’s greenhouse gas emissions.

\textbf{Ozone-depleting substances drop to 2 percent of 1986 values}

Globally, the production of ozone-depleting substances has decreased to 2 percent of their 1986 values. Atmospheric ozone is important as it absorbs harmful ultraviolet rays from the sun. Emissions from chlorofluorocarbons and other ozone-depleting substances have caused a hole to open in the ozone layer above Antarctica. While this ‘ozone hole’ has little effect on UV levels in our atmosphere, filaments of ozone-poor air can sometimes pass overhead when the ozone hole breaks up in spring.

The Montreal Protocol on Substances that Deplete the Ozone Layer, agreed in 1987, led to a decrease in the production of ozone-depleting substances. This decrease is expected to reduce the size of the ozone hole to pre-1980 levels by around the middle of this century (United Nations Environment Programme Ozone Secretariat, nd).

For more detail see Environmental indicators Te taiao Aotearoa: Global emissions of ozone-depleting substances.

\subsection{Natural pressures}
This section presents information on two natural pressures on our climate: oceans and climate oscillations.

\textbf{Oceans influence our climate}

The oceans influence our climate through the exchange of heat and moisture between the ocean and the atmosphere. For example, the oceans provide much of the heat that drives tropical cyclones, which are a significant influence on New Zealand’s weather.

The oceans also moderate the effects of climate change by absorbing carbon dioxide and heat from the atmosphere. However, absorbing carbon dioxide makes oceans more acidic (Ciais et al, 2013).

For more information see: Marine chapter.

For moredetail see Environmental indicators Te taiao Aotearoa: Coastal sea-surface temperature, Oceanic sea-surface temperature, and Ocean acidification.

\textbf{Climate oscillations can affect our climate}

Climate oscillations are natural variations in the world’s climate patterns. Three of these have a particular influence on New Zealand’s weather patterns – the El Niño Southern Oscillation, the Interdecadal Pacific Oscillation, and the Southern Annular Mode.

The El Niño and La Niña phases of the El Niño Southern Oscillation influence New Zealand’s temperature and rainfall. El Niño can lead to droughts in the east of New Zealand, while La Niña can lead to flooding in the north (NIWA, 2007).

For more detail see Environmental indicators Te taiao Aotearoa: Climate oscillations.

\section{The state of our atmosphere and climate}

This section sets out the state of our atmosphere and climate. It presents information on greenhouse gases, ozone and ultraviolet intensity, temperature, rainfall, and extreme weather.

\subsection{Greenhouse gases, ozone, and ultraviolet intensity}

This section presents information on greenhouse gases, ozone concentrations, and ultraviolet (UV) intensity in our atmosphere.

\textbf{Greenhouse gases are increasing}

Greenhouse gases in the atmosphere absorb heat radiating from the Earth’s surface causing warmer temperatures. The main greenhouse gases generated by human activities are carbon dioxide, methane, nitrous oxide, and carbon monoxide. Concentrations of these gases are measured by NIWA at Baring Head near Wellington. Carbon dioxide has the greatest impact over the long term because it persists in the atmosphere longer than the other greenhouse gases. However, the other greenhouse gases have a significant impact in the short term, because they absorb much more heat per kilogram than carbon dioxide.

Carbon dioxide concentrations measured over New Zealand increased 21 percent since measurements began in 1972 (see figure 15). This is an increase of about 1.6 parts per million a year (0.5 percent).

%Figure 15
%Note: Baring Head is near Wellington. It is a site that is less likely to be influenced by local sources of pollution. These observations are made only when the wind is blowing from the south and away from any likely local sources of gas emissions. This gives a measure representative of the concentrations over the Southern Ocean. Data are unavailable for some periods.
%Read a description of this image

Methane and nitrous oxide concentrations also increased significantly since they were first measured in 1989 and 1996, respectively. On the other hand, carbon monoxide concentrations decreased significantly since first measured in 2000. Globally, technological improvements and stricter controls on vehicle and industrial emissions are leading to lower emissions of carbon monoxide (Mikaloff-Fletcher \& Nichol, 2014).

The measurements taken in New Zealand are consistent with other observations around the globe. This is expected, because these gases get mixed around the globe by weather systems (Mikaloff-Fletcher \& Nichol, 2014).

For more detail see Environmental indicators Te taiao Aotearoa: Greenhouse gas concentrations.

\textbf{Ozone concentrations over New Zealand are comparable to countries of similar latitudes}

The thickness of the ozone layer is significant because ozone absorbs UV light from the sun and acts as a filter against sunburn and other damage (eg to plastic and fabric). Changes in ozone concentrations can also influence the risk of skin cancer.

The amount of ozone in an atmospheric column is measured in Dobson units, where 1 Dobson unit corresponds to the amount of ozone required to form a 0.01 millimetre layer of pure ozone. The global average ozone is 300 Dobson units (Liley \& McKenzie, 2007), slightly lower than the average ozone over New Zealand (307 Dobson units). Over New Zealand, the ozone layer ranges from about 275 Dobson units in autumn to 345 Dobson units in spring (see figure 16). Ozone varies by around 15 Dobson units from day to day (around 5 percent). Ozone concentrations over New Zealand are similar to other places around the world at similar latitudes.

%Figure 16
%Note: Maximum, minimum, and average ozone concentrations are shown.
%Read a description of this image

A small but discernible downward trend is evident in ozone concentrations over New Zealand. However, this may only have had a small impact on our UV levels.

For more detail see Environmental indicators Te taiao Aotearoa: Ozone concentrations.

The ozone hole only affects New Zealand’s ozone and UV levels when it breaks up in spring – when it can send plumes of air with lower ozone concentrations over the country. Even then, the ozone above New Zealand drops only about 5 percent, which is within the normal daily variation.

For more detail see Environmental indicators Te taiao Aotearoa: Ozone hole.

\textbf{New Zealand’s UV intensity is relatively high}

Peak UV intensities in New Zealand are about 40 percent greater than at comparable latitudes in the Northern Hemisphere. The strength of UV light is measured using a solar UV index (UVI) or sun index. A UVI above 11 is extreme.

In New Zealand, UV indexes vary day to day, and showed no consistent trend across the five monitored sites between 1981 and 2014. In summer, the UVI often exceeds 11. For example, the northernmost site, Leigh, had an average of 73 days with a UVI greater than 11. In winter, the UVI can be as low as 1 even on clear days, mainly because the sun is low in the sky.

For more detail see Environmental indicators Te taiao Aotearoa: UV intensity.

\subsection{Temperature, rainfall, and extreme weather}

This section presents information on average temperatures, rainfall, and extreme weather. We also collect data on sunshine hours, but that is not discussed here (for information on sunshine hours see Environmental indicators Te taiao Aotearoa: Sunshine hours).

\textbf{Average temperatures have risen}

The average annual temperature in New Zealand increased by about 0.9 degrees Celsius over the last century (a statistically significant trend), slightly lower than the global average land-based temperature increase of around 1–1.2 degrees Celsius (NIWA, 2015) (see figure 17).

The Intergovernmental Panel on Climate Change concluded that global warming is beyond doubt, and it is extremely likely that human-caused pressures such as the burning of fossil fuels has been the primary cause since 1950 (IPCC, 2013).

%Figure 17
%Note: Data from seven sites around the country, adjusted for changes in site location. The plot shows the difference from the 1981–2010 average (known as ‘normal’). Climatologists use the term ‘normal’ to refer to an average of the weather across a 30-year period.
%Read a description of this image

New Zealand’s average temperature, as estimated from seven monitored sites, fluctuates by about 0.4 degrees Celsius from year to year. This is a natural variation not directly linked to human-caused pressures. These year-to-year changes can be linked in part to climate oscillations such as the El Niño Southern Oscillation.

The increase in temperature is also reflected in ‘growing degree days’. Growing degree days is the total days over a year in which the mean daily temperature is higher than a base value (10 degrees Celsius).

Of the 29 sites assessed, 18 showed upward trends, with a median increase across all sites of 0.22 growing degree days per year.

For more detail see Environmental indicators Te taiao Aotearoa: National temperature time series and Growing degree days.

\textbf{Rainfall is highly variable with no clear trend}

Rainfall is highly variable both geographically and from year to year, so we have been unable to determine any long-term trends. However, it is expected that climate change will influence rainfall (and other forms of precipitation) over the longer term.

Geographically, annual average rainfall varies from less than 500 millimetres in dry years in places like Central Otago, to over 4,000 millimetres in mountainous areas such as the Tararua Range, and even more in the Southern Alps (see figure 18).

For more detail see Environmental indicators Te taiao Aotearoa: Annual rainfall.

%Figure 18
%Read a description of this image

\textbf{Fewer days with gale-force winds but no clear trends in other extreme weather}

This section presents information about three common elements of severe weather events: annual three-day rainfall maximums, the number of days with wind gusts above gale force, and lightning frequency.

Three-day rainfall data capture rainfall events likely to cause widespread flooding in affected areas. These data show the greatest rainfall total over any three-day period for each year, at each site. The data vary from year to year, so we cannot identify any statistically significant trends. While these data (see figure 19) show the maximum three-day rain totals for each year, they do not show the frequency of heavy rainfall events in a year.

%Figure 19
%Note: Data was unavailable for Wellington in 1993.
%Read a description of this image

We estimate damaging wind by the number of days a year wind gusts exceed gale force (61 kilometres an hour). The data show that Auckland, Wellington, and Christchurch regularly experience gale-force gusts, but Wellington is particularly prone to gale-force winds. Since 1975, the occurrence of potentially damaging wind has decreased in Wellington. Monthly (but incomplete) data for Christchurch and Auckland also suggest a smaller decrease (see figure 20). It is not clear whether this decrease is part of a natural cycle or is influenced by climate change. Studies predict global climate change may increase the frequency of damaging wind events in winter in almost all areas in New Zealand, and decrease the frequency in summer (Mullan et al, 2011).

%Figure 20:
%Note: Two of the three selected sites have missing data for some years. Gusts above gale force – above 33 knots, approximately 61 km/h.
%Read a description of this image

Annually, there are about 190,000 lightning strikes on the New Zealand land mass and over coastal seas. This is not frequent by international standards. Lightning strikes are most frequent on the West Coast of the South Island. The data show a high degree of variability over the 14-year period from September 2000 to December 2014. However, thunderstorms, and therefore lightning, are expected to increase in frequency and intensity as a result of climate change (Mullan et al, 2011).

For more detail see Environmental indicators Te taiao Aotearoa: Annual maximum three-day rainfall, Occurrence of potentially damaging wind, and Lightning.

\section{Glossary}

\section{References}

\section{Copyright and disclaimer}


\end{document}
\grid
